% The first command in your LaTeX source must be the \documentclass command.

\documentclass[sigconf]{acmart}
 % Do not change for ICTIR'19

\settopmatter{printacmref=true}
  % mandatory for ICTIR'19

\fancyhead{}
  % do not delete this code.
\usepackage{balance}
  % for creating a balanced last page (usually last page with references)
 %activate todo's
\newcommand{\todo}[1]{\textcolor{red}{TODO: #1}\PackageWarning{TODO:}{#1!}}
% deactivate todos
%\newcommand{\todo}[1]{ \PackageWarning{TODO:}{#1!}}

% defining the \BibTeX command - from Oren Patashnik's original BibTeX documentation.
\def\BibTeX{{\rm B\kern-.05em{\sc i\kern-.025em b}\kern-.08emT\kern-.1667em\lower.7ex\hbox{E}\kern-.125emX}}
    
% Rights management information. 
% This information is sent to you when you complete the rights form.
% These commands have SAMPLE values in them; it is your responsibility as an author to replace
% the commands and values with those provided to you when you complete the rights form.
%
% These commands are for a PROCEEDINGS abstract or paper.

\copyrightyear{2019} 
\acmYear{2019} 
\acmConference[PredictGIS 2019]{ACM SIGSPATIAL Workshop on Prediction of Human Mobility}{November 5, 2019}{Chicago, IL, USA}
\acmBooktitle{The 2019 ACM SIGSPATIAL Workshop on Prediction of Human Mobility (Predict GIS '19), November 5, 2019, Chicago, IL, USA}



% Submission ID. 
% Use this when submitting an article to a sponsored event. You'll receive a unique submission ID from the organizers
% of the event, and this ID should be used as the parameter to this command.
\acmSubmissionID{ictir094s}


% end of the preamble, start of the body of the document source.

\begin{document}

\fancyhead{}
  % do not delete this code.


% The "title" command has an optional parameter, allowing the author to define a "short title" to be used in page headers.
\title{%
  Patterns of Origin Destination Distributions \\
  \large Rules Mining using Massive GPS Trajectory Data}



%
% By default, the full list of authors will be used in the page headers. Often, this list is too long, and will overlap
% other information printed in the page headers. This command allows the author to define a more concise list
% of authors' names for this purpose.
%\renewcommand{\shortauthors}{Chatterjee and Dietz}

%
% The abstract is a short summary of the work to be presented in the article.
\begin{abstract}
Understanding the spatio-temporal distribution of road traffic conditions is imperative to the design of suitable countermeasures for congestion. The amount of crowd sourced data from private aggregators is rapidly expanding to include data sources that were previously not available; this data can assist in developing new approaches and algorithms to understand travel patterns and origin-destination (O-D) distributions. We acquired four months (February, June, July, and October of 2015) of INRIX waypoint data that includes vehicle trips from various sources. To determine the association between demographic, economic, and land use information and O-D patterns, we acquired Census block group level data from American Community Survey (ACS), and block level economic data from Longitudinal Employer-Household Dynamics (LEHD). We provided insights on the relationship between demographic information and O-D patterns by Census spatial unit block group. Based on multi-source data, we used classification-based association rules mining to determine key association patterns. 
\end{abstract}

%
% The code below is generated by the tool at http://dl.acm.org/ccs.cfm.
% Please copy and paste the code instead of the example below.
%
%\begin{CCSXML}
%<ccs2012>
% <concept>
%<concept_id>10002951.10003317.10003338</concept_id>
%<concept_desc>Information systems~Retrieval models and ranking</concept_desc>
%<concept_significance>500</concept_significance>
%</concept>
%</ccs2012>
%<concept>
%<concept_id>10002951.10003317.10003325</concept_id>
%<concept_desc>Information systems~Information retrieval query processing</concept_desc>
%<concept_significance>300</concept_significance>
%</concept>
%\end{CCSXML}
%
%\ccsdesc[500]{Information systems~Retrieval models and ranking}
%\ccsdesc[300]{Information systems~Information retrieval query processing}
%

%
% Keywords. The author(s) should pick words that accurately describe the work being
% presented. Separate the keywords with commas.
% \keywords{joint query-entity-passage features, entity context neighbors, entity salience, entity context document}


%
% This command processes the author and affiliation and title information and builds
% the first part of the formatted document.
\maketitle

\section{Introduction}
\label{sec:Introduction}

Roadside Interview Survey (RSI) has been considered as the common method to determine the measures of roadway origin-destinations (O-D). The crowd sourced data from the private companies can assist in developing new algorithms to understand O-D measures. Although these data sources suffer from many limitations, they are proving valuable insights in determining many traffic operational exposures. Many studies performed analysis on the O-D patterns using different regression models. In general, regression models ignore subgroup or clusters in the data sets. Rules based models can identify the subgroup effects from complex and large data sets without imposing any prior assumptions. 

We acquired four months (February, June, July, and October of 2015) of INRIX waypoint data that includes vehicle trips from various data providers. We provided insights on the relationship between demographic information and O-D patterns. To determine the relationship between O-D patterns and other area specific variables, we acquired Census block group level data from American Community Survey (ACS), and block level economic data from Longitudinal Employer-Household Dynamics (LEHD). We performed our analysis by keeping Census block group as the spatial unit. We applied classification-based association rules mining to determine significant rules.

\section{Literature Review}
\label{sec:related work}
%\paragraph{\textbf{Sentence Retrieval}}
Many recent studies have aimed to determine origin destination patterns using crowd sourced data. Some key studies in this area are briefly described in this section. \\

In a study conducted by Sana et al. [1], researchers used information from the Google Aggregated and Anonymized Trips (AAT) to develop a machine learning model and generate the San Francisco Bay Area hourly O-D demand matrices. They found that the developed model could effectively predict dynamic O-D person trip matrices by using both existing and future versions of AAT information. In another study, Ma et al. [2] developed a data-driven structure to estimate daily dynamic O-D. To accomplish this, researchers used high-granular traffic frequency and speed data spanning over many years. The developed framework employed t-Distributed Stochastic Neighbor Embedding (t-SNE) and k-means techniques to statistically cluster regular traffic data into typical traffic models. Fan et al. [3] conducted a study in Guangzhou City, China; they developed an O-D assessment methodology for systematic transit travelers. The researchers used smart card bus transportation data to improve the trip-chain O-D estimation algorithms. The results of the study were useful for the real-world work associated with the O-D estimation. In another study, Ge et al. [4] used aggregated data of mobile phone traces to estimate work-related trips and develop a method to estimate O-D matrices based on maximum entropy principle. The researchers calculated the trip production and attraction by using non-linear programming problem; they then used a matrix fitting problem to distribute trips to each O-D pair. Furthermore, two recent studies used Maryland INRIX waypoint data; one study estimated vehicle miles traveled [5] and another determined the reliability of truck drivers’ routing decisions [6]. 

%\paragraph{\textbf{Entity Relation Explanation}}
 \begin{figure*}
    \centering
    \includegraphics[width=\textwidth]{Fig3a.JPG}
    \caption{O-D patterns by Month}
    \label{fig_dimensions}
 \end{figure*}
 
 
%\paragraph{\textbf{Entity Relation Explanation}}
 \begin{figure*}
    \centering
    \includegraphics[width=\textwidth]{Fig4.JPG}
    \caption{O-D patterns by Day}
    \label{fig_dimensions}
 \end{figure*}

\section{Approach}
\label{sec:approach}
%  not needed: \subsection{Overview}
%\label{subsec:overview}
Given a ranked list of entities for a query, we seek to embellish it with passages which would explain to the user why the entity is relevant to the query. We call the entities in the ranking as target entities. We only try to predict support passages for target entities which are also relevant (according to the entity ground truth for the query). To model this joint relevance for \textit{both} the query and the entity, we use the co-occurring entities with a given entity. 

\subsection{Data Sources}
\label{subsec:features}
The research team used three separate databases to perform the analysis:
\begin{itemize}
    \item Maryland INRIX waypoint trip data
    \item Census block group level ACS 2013-2017 data
    \item Census block level LEHD 2015 data
\end{itemize}
\subsubsection{Maryland INRIX trip data. }
\label{subsubsec:features:1}
The research team members collected Maryland INRIX trip data to perform the analysis. The acquired data for four months (February, June, July and October 2015) contain three types of monthly files in comma separated value (CSV) format:
\begin{itemize}
    \item TripRecordsReport (trip data)
    \item TripRecordsReportWaypoints (waypoint data)
    \item TripsRecordsReportProviderDetails (information on trip data providers)
\end{itemize}
The acquired dataset contains:
\begin{itemize}
    \item 19,690,402 trip information
    \item 1,376,720,203 waypoint (geospatial point recorded by GPS devices that represents the location of the recorded vehicle) information
    \item 5,451,095 unique device identifications
    \item 148 data providers (45 providers)
    \begin{itemize}
        \item 3 providers account for 52 percent of trips
        \item 18 providers account for 99 percent of trips
    \end{itemize}    
    \item Four types of vehicle driving profiles: 1) consumer vehicles, 2) taxi/shuttle/town car service fleets, 3) local delivery fleets, and 4) for-hire/private trucking fleets.
    \item Three vehicle weight classes: 1) light duty truck/passenger vehicle (0 to 14000 lbs.), 2) medium duty truck/vans (14001 to 26000 lbs.), and 3) heavy-duty truck (greater than 26000 lbs.).
\end{itemize}


\subsubsection{ACS Data}
\label{subsubsec:feature:2}
The ACS data, acquired from the U.S. Census Bureau's Decennial Census Program, provide details of demographic, social, housing, and economic estimates for different spatial area units including Census tract, and block groups. The research team used ACS five-year (2013-2017) estimates for Maryland. The research team collected a wider list of variables in the preliminary analysis.  

\subsubsection{LEHD Data. }
\label{subsubsec:feature:3}
The research team also used the Census block data of LEHD Origin-Destination Employment Statistics (LODES) 2015 data. The LODES files contain data for Residential Area Characteristics (RAC) and Work Area Characteristics (WAC). The block level data are merged into block group level to perform the analysis.

\subsection{Data Integration}
\label{subsec:features}
The research team used three separate databases to perform the analysis:
\begin{itemize}
    \item The team members acquired 2013-2017 ACS data for Maryland from the U.S. Census. ACS data contains different tables such as age, gender, income, and household. The research team compiled the data for the key demographic and relevant data. 
    \item The team members collected block level 2015 LEHD data for Maryland. The block level RAC and WAC data are merged into block group level. 
    \item The research team used QGIS tool to spatially merge origin and destination data with census block groups. Later several separate databases were developed based on the following:
    \begin{itemize}
        \item Monthly O-D data by block groups
        \item Daily O-D data by block groups    
        \item Hourly O-D data by block groups
        \item O-D data by vehicle types  
    \end{itemize}
\end{itemize}

%\paragraph{\textbf{Entity Relation Explanation}}
\begin{figure}
    \includegraphics[width=\columnwidth]{Fig2a.JPG}
    \caption{Hourly O-D patterns by consumer vehicles}
    \label{fig_dimensions}
\end{figure}

%\paragraph{\textbf{Entity Relation Explanation}}
\begin{figure}
    \includegraphics[width=\columnwidth]{Fig1a.JPG}
    \caption{Hourly O-D patterns by taxi and shuttles}
    \label{fig_dimensions}
\end{figure}
 

\section{Evaluation}
\label{sec:Evaluation}
\subsection{Research Questions}
\label{subsec:research questions}
We study the extent to which the following components in our approach affect the quality of retrieved support passages:

\begin{itemize}
\item[\textbf{RQ1}] What is the effect of contextual entities?
\item[\textbf{RQ2}] What is the effect of entity salience? 
\item[\textbf{RQ3}] Is it better to expand queries with entities or words?
\end{itemize}

%Our goal is to study the following research questions.\\
%\textbf{RQ1} To what extent do entities in context affect the support passage retrieval?  \\\\
%\textbf{RQ2} To what extent does entity salience affect the support passage retrieval?  \\
%\textbf{RQ3} Are contextual entities more informative than contextual words? 


\subsection{O-D Distribution}
\label{subsec:Evaluation Paradigm}
Figure 1 illustrates the origin and destination trips by block groups. A lighter color in a certain area indicates the lower number of trips, while darker colors indicate a higher number of trips per block group.

It is important to determine the key O-D matrices for different scenarios. Three different temporal patterns were used for analysis: 1) all 24 hours, 2) morning peak (6-10 am, Mon-Fri), and 3) evening peak (4-8 pm, Mon-Fri). Figure 2 indicates that the top O-D generators vary by different temporal slices. 

A chord diagram, resourceful in comparing the similarities and patterns within a dataset, illustrates the interrelationships between individuals (for this study, each block group is considered as individual spatial unit). The associations between individual block groups are used in displaying commonality of information or interest. Nodes are arranged in a circular form, with the association between points connected to each other either with arcs or curves. The assigned values to each of the connection are represented proportionally by the size of each arc. The color is used in grouping the data into different categories that aid in making comparisons and distinguishing groups. Figure 3 indicates that the in-betweenness between the O-D pairs is significant. \\


\section{Results}
\label{subsubsec: reults}
\subsection{Rules Mining}
\label{subsec:Rules Mining}
This study used classification-based association rules to develop the rules associated with O-D patterns. The association rule can be represented as $Antecedent (A) \rightarrow Consequent (B)$, where both

of them are disjoint itemsets. Here, $Support\ of\ antecedent,\ S\left(A\right)=\frac{\sigma(A)}{N};\ Support\ of\ consequent,\ S\left(B\right)=\frac{\sigma(B)}{N};$ and Support of the rule  can be written as 
$Suppport\ of\ rule\ or\ S\left(A\rightarrow B\right)=\frac{\sigma(A\cap B)}{N};$
The measure of reliability for a generated rule is known as confidence ($C\left(A\rightarrow B\right)=\frac{S\left(A\rightarrow B\right)}{S(A)}$).  

The lift ($L\left(A\rightarrow B\right)=\frac{S\left(A\rightarrow B\right)}{S\left(A\right).S(B)}$) is a measure that represents the ratio of confidence and expected confidence. A lift value greater than one shows positive interdependence between A and B, while a value smaller than one refers negative interdependence, and a value of one signifies independence.

The research team selected the significant variables by performing random forest algorithm. The selected variables (block group level) for model development include average O-D measure, total population (Popu), households (HH), households with family (HH\_F), total WAC jobs (Total\_Jobs\_WAC), and household median income (HH\_MedInc). The top 15 rules with high lift values are listed in Table 1. Average O-D measures per block group are divided into five classes based on the quantile percentages: TQ= 1 [1-20 percent], TQ= 2 [21-40 percent], TQ= 3 [41-60 percent], TQ= 4 [61-80 percent], and TQ=5 [81-100 percent]. For example, TQ=5 indicates the block groups with top 20 percent percent of the O-D trips. The findings show that higher number of populations, WAC, and households are associated with higher   number of O-D trips. The rules provide several breakpoints of the variable clusters to determine the top rules. 	



\begin{table*}[t]
  \centering
  \caption{Top 40 Rules.}
  \label{tab:results}
\begin{tabular}{|l|l|l|l|l|l|}
\hline
Antece.                                                                                & Conse. & S    & C    & L    & Counts \\ \hline
Total\_Jobs\_WAC=(\textgreater 3778.5{]}, HH\_F=(386.5; 659.5{]}                       & TQ=5   & 0.01 & 1    & 5    & 35     \\ \hline
Total\_Jobs\_WAC=(\textgreater 3778.5{]}, Popu=(1366.5; 1978.5{]}                      & TQ=5   & 0.01 & 1    & 5    & 34     \\ \hline
Total\_Jobs\_WAC=(\textgreater 3778.5{]}, Popu=(1978.5; 3311.5{]}                      & TQ=5   & 0.01 & 1    & 5    & 31     \\ \hline
Total\_Jobs\_WAC=(\textgreater 3778.5{]}, HH\_MedInc=(\textgreater 98467{]}            & TQ=5   & 0.01 & 1    & 5    & 28     \\ \hline
Total\_Jobs\_WAC=(\textgreater 3778.5{]}, HH\_MedInc=(53774; 98467{]}                  & TQ=5   & 0.02 & 0.98 & 4.92 & 61     \\ \hline
Total\_Jobs\_WAC=(\textgreater 3778.5{]}                                               & TQ=5   & 0.03 & 0.96 & 4.82 & 109    \\ \hline
Popu=(\textgreater 3311.5{]}, HH=(\textgreater 1252.5{]}, HH\_F=(\textgreater 934.5{]} & TQ=5   & 0.01 & 0.91 & 4.53 & 29     \\ \hline
Total\_Jobs\_WAC={[}\textless 21.5{]}, HH={[}\textless 264.5{]}                        & TQ=1   & 0.02 & 0.9  & 4.52 & 75     \\ \hline
Total\_Jobs\_WAC={[}\textless 21.5{]}, HH\_F={[}\textless 230.5{]}                     & TQ=1   & 0.04 & 0.88 & 4.41 & 157    \\ \hline
Total\_Jobs\_WAC=(833; 3778.5{]}, HH=(\textgreater 1252.5{]}                           & TQ=5   & 0.01 & 0.88 & 4.4  & 22     \\ \hline
HH=(\textgreater 1252.5{]}, HH\_MedInc=(\textgreater 98467{]}                          & TQ=5   & 0.01 & 0.88 & 4.38 & 21     \\ \hline
Total\_Jobs\_WAC={[}\textless 21.5{]}, HH\_MedInc={[}\textless 53774{]}                & TQ=1   & 0.04 & 0.82 & 4.08 & 137    \\ \hline
Popu=(\textgreater 3311.5{]}, HH\_F=(\textgreater 934.5{]}                             & TQ=5   & 0.01 & 0.81 & 4.05 & 34     \\ \hline
Total\_Jobs\_WAC={[}\textless 21.5{]}, Popu=(623.5; 1366.5{]}                          & TQ=1   & 0.05 & 0.81 & 4.03 & 171    \\ \hline
Total\_Jobs\_WAC=(833; 3778.5{]}, HH\_F=(659.5; 934.5{]}                               & TQ=5   & 0.01 & 0.8  & 4    & 52     \\ \hline
Total\_Jobs\_WAC={[}\textless 21.5{]}, HH=(264.5; 477.5{]}                             & TQ=1   & 0.04 & 0.8  & 3.99 & 155    \\ \hline
Total\_Jobs\_WAC=(833; 3778.5{]}, HH=(972.5; 1252.5{]}                                 & TQ=5   & 0.02 & 0.79 & 3.97 & 58     \\ \hline
Popu=(\textgreater 3311.5{]}, HH=(\textgreater 1252.5{]}                               & TQ=5   & 0.01 & 0.79 & 3.95 & 45     \\ \hline
Total\_Jobs\_WAC=(21.5; 61.5{]}, Popu={[}\textless 623.5{]}                            & TQ=1   & 0.01 & 0.78 & 3.91 & 36     \\ \hline
Total\_Jobs\_WAC=(833; 3778.5{]}, HH\_MedInc=(\textgreater 98467{]}                    & TQ=5   & 0.03 & 0.78 & 3.89 & 102    \\ \hline
Total\_Jobs\_WAC=(61.5; 120.5{]}, Popu={[}\textless 623.5{]}                           & TQ=1   & 0.01 & 0.78 & 3.89 & 28     \\ \hline
Total\_Jobs\_WAC=(833; 3778.5{]}, Popu=(\textgreater 3311.5{]}                         & TQ=5   & 0.01 & 0.76 & 3.82 & 26     \\ \hline
HH\_F=(\textgreater 934.5{]}                                                           & TQ=5   & 0.01 & 0.76 & 3.8  & 35     \\ \hline
Total\_Jobs\_WAC=(833; 3778.5{]}, Popu=(1978.5; 3311.5{]}                              & TQ=5   & 0.03 & 0.76 & 3.8  & 123    \\ \hline
Total\_Jobs\_WAC=(833; 3778.5{]}, HH\_F=(386.5; 659.5{]}                               & TQ=5   & 0.04 & 0.75 & 3.77 & 141    \\ \hline
Popu=(\textgreater 3311.5{]}, HH\_MedInc=(\textgreater 98467{]}                        & TQ=5   & 0.01 & 0.73 & 3.66 & 30     \\ \hline
HH=(\textgreater 1252.5{]}                                                             & TQ=5   & 0.02 & 0.72 & 3.59 & 61     \\ \hline
Total\_Jobs\_WAC={[}\textless 21.5{]}                                                  & TQ=1   & 0.07 & 0.72 & 3.58 & 275    \\ \hline
Total\_Jobs\_WAC=(833; 3778.5{]}, HH\_MedInc=(53774; 98467{]}                          & TQ=5   & 0.05 & 0.71 & 3.55 & 179    \\ \hline
Total\_Jobs\_WAC=(833; 3778.5{]}, HH=(609.5; 972.5{]}                                  & TQ=5   & 0.03 & 0.71 & 3.54 & 131    \\ \hline
Total\_Jobs\_WAC=(21.5; 61.5{]}, HH={[}\textless 264.5{]}                              & TQ=1   & 0.01 & 0.71 & 3.54 & 46     \\ \hline
Total\_Jobs\_WAC=(833; 3778.5{]}, Popu=(1366.5; 1978.5{]}                              & TQ=5   & 0.03 & 0.71 & 3.54 & 99     \\ \hline
Total\_Jobs\_WAC=(833; 3778.5{]}                                                       & TQ=5   & 0.1  & 0.69 & 3.45 & 362    \\ \hline
Total\_Jobs\_WAC=(61.5; 120.5{]}, HH={[}\textless 264.5{]}                             & TQ=1   & 0.01 & 0.67 & 3.33 & 30     \\ \hline
Popu={[}\textless 623.5{]}, HH\_MedInc={[}\textless 53774{]}                           & TQ=1   & 0.02 & 0.66 & 3.31 & 94     \\ \hline
Total\_Jobs\_WAC=(21.5; 61.5{]}, HH\_F={[}\textless 230.5{]}                           & TQ=1   & 0.02 & 0.64 & 3.18 & 89     \\ \hline
Popu={[}\textless 623.5{]}, HH={[}\textless 264.5{]}                                   & TQ=1   & 0.03 & 0.62 & 3.08 & 122    \\ \hline
HH={[}\textless 264.5{]}, HH\_MedInc={[}\textless 53774{]}                             & TQ=1   & 0.03 & 0.6  & 3.01 & 98     \\ \hline
Total\_Jobs\_WAC=(405.5; 833{]}, HH=(972.5; 1252.5{]}                                  & TQ=4   & 0.01 & 0.58 & 2.91 & 25     \\ \hline
HH=(972.5; 1252.5{]}, HH\_MedInc=(\textgreater 98467{]}                                & TQ=5   & 0.01 & 0.56 & 2.81 & 32     \\ \hline
\end{tabular}
\end{table*}



The most interesting results are presented in Table \ref{tab:results}. Due to space contraints other results were moved to the online appendix for this paper. Below, we discuss each of the research questions presented in Section \ref{subsec:research questions}.

\textbf{RQ1 (contextual entities):} We observe in Table \ref{tab:results} that ranking passages with contextual entities (ECN) achieves a MAP of 0.31 and query expansion with entities (QEE) a MAP of 0.30. This is a significant improvement over the two baselines which have a MAP of 0.09 and 0.07 respectively. This demonstrates the benefit of contextual entities. See Section \ref{subsec: ablation_study} for more discussion.

\textbf{RQ2 (entity salience):} We observe in Table \ref{tab:results} that the two methods which rank passages using entity salience perform the worst. This is also the case when combined with other features using learning-to-rank. We manually confirmed that SWAT correctly identifies salient and non-salient entities.  However, only few retrieved entities have a passage with a salient mention. While entities with salient passages are often relevant, a majority (95\%) of retrieved entities do not have a passage with a salient mention in the candidate pool. Since the salience feature is only applicable to very few entities, it only has a limited impact on the overall result. 

\textbf{RQ3 (query expansion):} We observe that among all methods which retrieve with contextual entities, the method using LMJM with RM3 performs the best with a MAP of 0.30. Note that here we use RM3 to expand the query with entities (see Section \ref{subsubsec:features:1}(\ref{en: Query expansion using entities from entity context documents})). Also, among all methods which retrieve with contextual words, the methods using LMJM with RM3 performs the best with a MAP of 0.35.  As we can see, retrieval with contextual words outperforms retrieval with contextual entities and although contextual entities achieve good results, they are unable to outperform contextual words.  Hence, contextual words are more informative than contextual entities. Another observation is that among BM25, LMDS and LMJM, retrieval with LMJM always gives the best performance for both contextual entities and contextual words. For more discussion, see Section \ref{subsec: ablation_study}.


\section{Conclusion and Future Work}
\label{sec: conclusion}
This work addresses the task of support passage retrieval to enrich entity rankings in response to a search query. We propose a joint query-entity-passage ranking method and present some initial results. In particular, we show that co-occurring entities are an important indicator of which passages might support an entity for a query. In contrast, retrieving passages using a compound query of the original query and the entity is not sufficient for the problem.

We also experiment with entity salience, and find that it is a highly informative feature when it is applicable, however 95\% of target entities do not have any passage with a salient mention in the candidate set. We thus identify a need for developing high-recall salience techniques that are applicable to a larger number of entities, as well as new indexing and retrieval methods that integrate entity salience in an early phase.




\bibliographystyle{ACM-Reference-Format}
\bibliography{bibliography}
\begin{enumerate}
\item Sana, B., J. Castiglione, D. Cooper, and D. Tischler. Using Google Passive Data and Machine Learning for Origin-Destination Demand Estimation. Transportation Research Record, Vol 2672, Issue 46, 2018, p. pp 73\textemdash82.
\item  Ma, W., and Z. Qian. Estimating Multi-Year 24/7 Origin-Destination Demand Using High-Granular Multi-Source Traffic Data. Transportation Research Part C: Emerging Technologies, Vol. 96, 2018, pp. 96\textemdash121. 
\item Fan., W, and Z. Chen. Estimation of origin-destination matrix and identification of user activities using public transit smart card data. Center for Advanced Multimodal Mobility Solutions and Education, 2018. 
\item Ge, Q., and F. Daisuke. Updating Origin–Destination Matrices with Aggregated Data of GPS Traces. Transportation Research Part C: Emerging Technologies, Vol. 69, 2016, pp. 291\textemdash312.
\item Fan, J., C. Fu, K. Stewart, and L. Zhang. Using big GPS trajectory data analytics for vehicle miles traveled estimation. Transportation Research Part C, Vol. 103, 2019, pp. 298 \textemdash 307. 
\item Kong, X., W. Eisele, Y. Zhang, and D. Cline. Evaluating the Impact of Real-time Mobility and Travel Time Reliability Information on Truck Drivers’ Routing Decisions. Transportation Research Record: Journal of the Transportation Research Board. Vol. 2672, 2018, pp. 164\textemdash172.
\end{enumerate}





\end{document}
